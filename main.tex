\documentclass{lmcs}

%% optional lists of keywords
\keywords{digital circuits,%
    symmetric traced monoidal categories, %
    denotational semantics, %
    operational semantics, %
    algebraic semantics, %
    string diagrams, %
    stream functions, %
    mealy machines}%

\usepackage{amsthm}
\usepackage{amsmath}
\usepackage{stmaryrd}

\usepackage{etoolbox}
\usepackage{xparse}
\usepackage{mleftright}

\usepackage{standalone}
\usepackage{wrapfig}
\usepackage{booktabs}
\usepackage{microtype}
\usepackage{multicol}
\usepackage{mathtools}

\usepackage{figures/tikzit}
\usepackage{tikz-cd}

\usepackage{bussproofs}
\usepackage{mathpartir}
\usepackage{hyperref}
\usepackage[capitalise]{cleveref}

\input{macros/letters}
\input{macros/sets}
\input{macros/category}
\input{macros/circuits}
\input{macros/streams}
\input{macros/graphs}
\input{macros/proofs}
\input{figures/circuits.tikzdefs}
\input{figures/circuits.tikzstyles}

\newcommand{\bgcolour}{white}

\begin{document}

\title[A Complete Theory of Digital Circuits]{%
    A Complete Theory of Digital Circuits: %
    Denotational, Operational, and Algebraic Semantics}%

% \titlecomment{{\lsuper*}OPTIONAL comment concerning the title, \eg,
%     if a variant or an extended abstract of the paper has appeared elsewhere.}
% \thanks{thanks, optional.}

\author[D.R.~Ghica]{Dan R.\ Ghica\lmcsorcid{0000-0002-4003-8893}}[a]
\author[G.~Kaye]{George Kaye\lmcsorcid{0000-0002-0515-4055}}[a]
\author[D.~Sprunger]{David Sprunger\lmcsorcid{0000-0002-2518-4710}}[b]


\address{University of Birmingham}
\email{d.r.ghica@bham.ac.uk, georgejkaye@gmail.com}

\address{Indiana State University}
\email{david.sprunger@indstate.edu}

\begin{abstract}
    Digital circuits, despite having been studied for nearly a century and used
    at scale for about half that time, have until recently evaded a fully
    compositional theoretical understanding, in which arbitrary circuits may be
    freely composed together without consulting their internals.
    Recent work remedied this theoretical shortcoming by showing how digital
    circuits can be presented compositionally as morphisms in a freely generated
    symmetric traced category.
    However, this was done informally; in this paper we refine and expand the
    previous work in several ways, culminating in the presentation of three
    sound and complete semantics for digital circuits: denotational, operational
    and algebraic.
    For the denotational semantics, we establish a correspondence between
    stream functions with certain properties and circuits constructed
    syntactically.
    For the operational semantics, we present the reductions required to model
    how a circuit processes a value, including the addition of a new reduction
    for eliminating non-delay-guarded feedback; this leads to an adequate notion
    of observational equivalence for digital circuits.
    Finally, we define a new family of equations for translating circuits into
    bisimilar circuits of a `normal form', leading to a complete algebraic
    semantics for sequential circuits.
\end{abstract}

\maketitle

\section{Introduction}

Bothe was awarded the 1954 Nobel Prize in physics for creating the electronic
\(\andgate\) gate in 1924.
In the ensuing decades, exponential improvements in digital technology have led
to the development of the defining technologies of the modern world,
It may therefore seem improbable that there are theoretical gaps remaining in
our mathematical and logical understanding of digital circuits.

To be more precise, by `digital circuits' we primarily understand
\emph{electronic} circuits: deterministic circuits with clear notions of
input and output and which work on discrete signals.
A classic example is that of logical gates and basic memory elements of known
and fixed delays, but there are alternatives such as CMOS transistors operating
in saturation mode.
What we do not attempt to handle are circuits operating on continuous signals
(such as amplifiers) or in continuous time (such as asynchronous circuits), nor
\emph{electrical} circuits of resistors and capacitors, which are quite
different~\cite{boisseau2022string}.

Our goal is to devise a \emph{fully compositional} model of digital circuits.
By `fully compositional' we mean that a larger circuit can be built from smaller
circuits and interconnecting wires without paying heed to the internal structure
of these smaller circuits.
Of course, composition comes naturally to digital circuits and is widely used
informally~\cite{gordon1982model}.
Unfortunately one runs into an obstacle when trying to formalise this notion
mathematically: electrical connections can be created that inadvertently
connect the output of some elementary gate back to
its input such that no memory elements are encountered along the path.
Such a path, called a `combinational feedback loop' (or `cycle'), is not handled
by established mathematical theories of digital circuits, so conventional
digital design and engineering reject such circuits.
To enforce this restriction, we need to always look inside circuits as we
compose them, ensuring that no illegal combinational feedback loops are created,
or resort to some `safe' subset of circuits~\cite{christensen2021wire}.
This represents a failure of compositionality: what we want to do is to compose
\emph{any} circuits constructed from a fixed set of components.

On general principle, we have reason to expect that a compositional theory of
digital circuits may lead to more streamlined methods of analysis and
verification, which, in time, may also lead to new applications.
\emph{Combinational} circuits, which model functions, have an obvious
compositional syntax~\cite{lafont2003algebraic}, but \emph{sequential} circuits,
which contain delay and feedback, are more subtle.
The first forays towards a fully compositional syntactic and
categorical account of circuits have been made
recently~\cite{ghica2016categorical,ghica2017diagrammatic}, but they do not
paint a fully formal and coherent picture.
This paper develops the informal presentation into a mathematically
rigorous framework.

Our first contribution is to give, for the first time, a sound and complete
denotational semantics to digital circuits in the domain of causal and monotone
stream functions.
The completeness result depends on a novel albeit straightforward lifting of
Mealy machines~\cite{mealy1955method} to act on alphabets with a lattice
structure, utilising a handy coalgebraic perspective~\cite{rutten2006algebraic}.
Using Mealy machines to give a semantic interpretation to digital circuits is an
established methodology~\cite{kohavi2009switching}, and here they act as a
`bridge' between the syntactic and the semantic domain, showing how existing
circuit methodologies are compatible with our rigorous mathematical framework.

The second contribution of the paper is to generalise and systematise previous
efforts~\cite{ghica2017diagrammatic} to formulate a graph-rewriting-based
operational semantics for digital circuits.
The novelty is a new reduction rule for eliminating non-delay-guarded feedback
using a version of the Kleene fixpoint theorem, thus solving the problem of
productivity that previous operational semantics only solve partially.
The denotational and operational semantics together achieve the long-standing
goal of creating a semantic theory of digital circuits using the same
methodology as programming languages.

The methodological `glue' that binds together the two approaches is a new sound
and complete algebraic semantics, the third and final contribution of the paper.
This approach replaces the previous ad-hoc way of introducing equations for
digital circuits based on raw intuitions with a systematic approach guided by
the denotational semantics.
The key technical result of this method is deriving pseudo-normal forms of
digital circuits.

Although the motivation of this work is foundational, there are some early
hints of possible exciting applications, such as for partial evaluation and
blackboxing.
These are not yet industrial-strength applications, but the simplicity and
power of the framework must hold a certain degree of appeal and promise.

\section{Syntax}

Our \emph{soup du jour} is that of
\emph{sequential synchronous digital circuits}
constructed from primitive components such as logic gates or transistors.
These circuits are \emph{sequential} as they have a notion of \emph{state}:
outputs can be impacted by inputs in previous cycles rather than solely those in
the current cycle, and \emph{synchronous} because their state changes in time
with some global clock.

\begin{rem}
    Digital (electric) circuits are not to be confused with \emph{electronic}
    circuits of switches and resistors.
    Essentially, the difference boils down to the difference between traced
    categories and compact closed categories: digital circuits have a clear
    notion of \emph{causality} whereas electronic circuits are \emph{relational}
    in nature.
    For a study of the latter, see \cite{boisseau2022string}.
\end{rem}

\subsection{Circuit signatures}

To construct circuits, we define a category in which a morphism
\(m \to n\) is a circuit with \(m\) inputs and \(n\) outputs.
Rather than restricting to any particular gate set, we parameterise a given
category of circuits over a \emph{circuit signature} containing details about
the available components.

\begin{defi}[Circuit signature, value, primitive symbol]
    A \emph{circuit signature} \(\circuitsignature\) is a tuple \((
    \values,
    \disconnected,
    \circuitsignaturegates,
    \circuitsignaturearity,
    \circuitsignaturecoarity
    )\) where \(\values\) is a finite set of \emph{values}, \(
    \disconnected \in \values
    \) is a \emph{disconnected} value, \(\circuitsignaturegates\) is a (usually
    finite) set of \emph{primitive symbols}, \(
    \morph{\circuitsignaturearity}{\circuitsignaturegates}{\nat}
    \) is an \emph{arity} function and \(
    \morph{\circuitsignaturecoarity}{\circuitsignaturegates}{\nat}
    \) is a \emph{coarity} function.
\end{defi}

A particularly important signature, and one which we will turn to for the
majority of examples in this thesis, is that of gate-level circuits.

\begin{exa}[Gate-level circuits]\label{ex:belnap-signature}
    The circuit signature for \emph{gate-level circuits} is \(
    \belnapsignature \coloneqq (
    \belnapvalues,
    \bot,
    \belnapgates,
    \belnaparity,
    \belnapcoarity
    )\), where \(
    \belnapvalues \coloneqq \{\bot, \belnapfalse, \belnaptrue, \top\}
    \), respectively representing \emph{no} signal, a \emph{false} signal, a
    \emph{true} signal and \emph{both} signals at once, \(
    \belnapgates \coloneqq \{\andgate,\orgate,\notgate\}
    \), \(
    \belnaparity \coloneqq
    \andgate \mapsto 2,
    \orgate \mapsto 2,
    \notgate \mapsto 1
    \) and \(
    \belnapcoarity \coloneqq - \mapsto 1,
    \)
\end{exa}

\begin{rem}
    Using four values may come as a surprise to those expecting the usual
    `true' and 'false' logical values.
    This logical system is an old idea going back to
    Belnap~\cite{belnap1977useful} who remarked that this is `how a computer
    should think'.
    Rather than just thinking about how much \emph{truth content} a value
    carries, the four value system adds a notion of \emph{information content}:
    the \(\bot\) value is no information at all (a disconnected wire), whereas
    the \(\top\) value is both true and false information at once
    (a short circuit).
\end{rem}

\subsection{Combinational circuits}

Before diving straight into sequential circuits, we will define a category of
\emph{combinational circuits}.
These are circuits with no state; they compute \emph{functions} of their inputs.

\begin{defi}[Combinational circuit generators]
    Given a circuit signature \(
    \circuitsignature = (
    \circuitsignaturevalues,
    \bullet,
    \circuitsignaturegates,
    \circuitsignaturearity,
    \circuitsignaturecoarity
    )
    \), let the set \(\generators[\ccirc{}]\) of
    \emph{combinational circuit generators} be defined as the set containing \(
    \iltikzfig{circuits/components/gates/gate}[gate=g,colour=comb,dom=\circuitsignaturearity(g),cod=\circuitsignaturecoarity(g)]
    \) for each \(g \in \circuitsignaturegates\),
    \(\iltikzfig{strings/structure/monoid/init}[colour=comb]\),
    \(\iltikzfig{strings/structure/comonoid/copy}[colour=comb]\),
    \(\iltikzfig{strings/structure/monoid/merge}[colour=comb]\), and
    \(\iltikzfig{strings/structure/comonoid/discard}[colour=comb]\).
    We write \(\ccircsigma\) for the freely generated PROP
    \(\smc{\generators[\ccirc{}]}\).
\end{defi}

Each primitive symbol \(p \in \circuitsignaturegates\) has a corresponding
generator in \(\ccircsigma\).
The remaining generators are \emph{structural} generators
for manipulating
wires: these are present regardless of the signature.
In order, they are for \emph{introducing} wires, \emph{forking}
wires, \emph{joining} wires and \emph{eliminating} wires.

\begin{exa}
    The gate generators of \(\ccirc{\belnapsignature}\) are \(
    \iltikzfig{circuits/components/gates/and}
    \), \(
    \iltikzfig{circuits/components/gates/or}
    \), and \(
    \iltikzfig{circuits/components/gates/not}
    \).
\end{exa}

When drawing circuits, the coloured backgrounds of generators will often be
omitted in the interests of clarity.
Since the category is freely generated, morphisms are defined by
juxtaposing the generators in a given signature sequentially or in parallel with
each other, the identity and the symmetry.
Arbitrary combinational circuit morphisms defined in this way are drawn as light
boxes \iltikzfig{strings/category/f}[box=f,colour=comb,dom=m,cod=n].

\begin{nota}\label{not:arbitrary-width-structure}
    The structural generators are only defined on single bits, but it is
    straightforward to define versions for arbitrary bit wires.
    Much like we often draw multiple wires as one
    (\cref{not:arbitrary-width-wires}), we can also draw these `thicker'
    constructs in a similar way to the single-bit versions:
    \begin{gather*}
        \iltikzfig{strings/structure/monoid/init}[colour=comb,obj=n]
        \quad
        \iltikzfig{strings/structure/comonoid/copy}[colour=comb,obj=n]
        \quad
        \iltikzfig{strings/structure/monoid/merge}[colour=comb,obj=n]
        \quad
        \iltikzfig{strings/structure/comonoid/discard}[colour=comb,obj=n]
    \end{gather*}
    These composite constructs are defined inductively over the width of the
    wires.

    \begin{center}
        \begin{minipage}{0.48\textwidth}
            \centering
            \(\iltikzfig{strings/structure/comonoid/copy-unit}[obj=0,colour=comb]
            \coloneqq
            \iltikzfig{strings/monoidal/empty}
            \)
            \quad
            \(
            \iltikzfig{strings/structure/comonoid/copy}[obj=n+1,colour=comb]
            \coloneqq
            \iltikzfig{strings/structure/comonoid/copy-construction}
            \)

            \vspace{1em}

            \(
            \iltikzfig{strings/structure/monoid/init-unit}[obj=0,colour=comb]
            \coloneqq
            \iltikzfig{strings/monoidal/empty}
            \)\quad\(
            \iltikzfig{strings/structure/monoid/init}[obj=n+1,colour=comb]
            \coloneqq
            \iltikzfig{strings/structure/monoid/init-construction}
            \)
        \end{minipage}
        \quad
        \begin{minipage}{0.48\textwidth}
            \centering
            \(\iltikzfig{strings/structure/monoid/merge-unit}[obj=0,colour=comb]
            \coloneqq
            \iltikzfig{strings/monoidal/empty}
            \)
            \quad
            \(
            \iltikzfig{strings/structure/monoid/merge}[obj=n+1,colour=comb]
            \coloneqq
            \iltikzfig{strings/structure/monoid/merge-construction}
            \)

            \vspace{1em}

            \(\iltikzfig{strings/structure/comonoid/discard-unit}[obj=0,colour=comb]
            \coloneqq
            \iltikzfig{strings/monoidal/empty}
            \)\quad\(
            \iltikzfig{strings/structure/comonoid/discard}[obj=n+1,colour=comb]
            \coloneqq
            \iltikzfig{strings/structure/comonoid/discard-construction}
            \)
        \end{minipage}
    \end{center}
\end{nota}

\begin{rem}
    As mentioned during \cref{not:arbitrary-width-wires}, wires of zero width
    are usually drawn as empty space; in a similar fashion the zero width fork,
    join, and elimination constructs can be drawn as empty space or using
    `faded' wires.
\end{rem}

\begin{exa}[More logic gates]
    The \(\andgate\), \(\orgate\), and \(\notgate\) gates are not the only logic
    gates used in circuit design.
    A \(\nandgate\) gate \(
    \iltikzfig{circuits/components/gates/nand}
    \) acts as the inverse of an \(\andgate\) gate: it
    outputs true if none of the inputs are true.
    Similarly, a \(\norgate\) gate \(
    \iltikzfig{circuits/components/gates/nor}
    \) is the inverse of an \(\orgate\) gate.
    These two gates can be constructed in terms of the primitive gates in
    \(\belnapsignature\):

    \[
        \iltikzfig{circuits/components/gates/nand}
        \coloneqq
        \iltikzfig{circuits/components/gates/nand-construction}
        \qquad
        \iltikzfig{circuits/components/gates/nor}
        \coloneqq
        \iltikzfig{circuits/components/gates/nor-construction}
    \]

    Another type of gate is the \(\xorgate\) gate \(
    \iltikzfig{circuits/components/gates/xor}
    \), which outputs true if and only if exactly one of the inputs is
    true.
    In \(\ccirc{\belnapsignature}\) this is constructed as \[
        \iltikzfig{circuits/components/gates/xor}
        \coloneqq
        \iltikzfig{circuits/components/gates/xor-construction-1}
        =
        \iltikzfig{circuits/components/gates/xor-construction-2}.
    \]
\end{exa}

\begin{exa}[Half adder]\label{ex:half-adder}
    The \(\xorgate\) gate is used in a classic combinational circuit known as a
    \emph{half adder}, the basic building block of circuit arithmetic.
    A half adder takes two inputs and computes their \emph{sum} modulo
    \(2\) and the resulting \emph{carry}.
    That is to say, \(0+0\) has sum \(0\) and carry \(0\), \(1+0\) and \(0+1\)
    have sum \(1\) and carry \(0\), and \(1+1\) has sum \(0\) and carry \(1\).

    The sum is computed using an \(\xorgate\) gate and the carry by an
    \(\andgate\) gate.
    The design of a half adder along with its construction in
    \(\ccirc{\belnapsignature}\) is shown below.
    \[
        \iltikzfig{circuits/examples/half-adder/circuit}
        \qquad
        \iltikzfig{circuits/examples/half-adder/circuit-ccirc}
        =
        \iltikzfig{circuits/examples/half-adder/circuit-ccirc-2}
    \]
\end{exa}

\subsection{Sequential circuits}

Combinational circuits compute functions of their inputs, but have no internal
state.
This is all very well for doing simple calculations, but for all but the most
simple of circuits we need to be able to have \emph{memory}.
As we have mentioned earlier, such circuits are called
\emph{sequential circuits}.

Circuits gain state with \emph{delay} and \emph{feedback}.
The latter means we need to move into a symmetric \emph{traced} monoidal category.

\begin{defi}[Sequential circuits]
    For a circuit signature \(\circuitsignature\) with value set \(\values\),
    let \(\scircsigma\) be the STMC freely generated over the generators of
    \(\ccircsigma\) along with new generators \(
    \iltikzfig{circuits/components/values/vs}[val=v]
    \) for each \(v \in \values \setminus \bullet\), and a generator \(
    \iltikzfig{circuits/components/waveforms/delay}
    \).
\end{defi}

The first set of generators are \emph{instantaneous values} for each value in
\(\values \setminus \bullet\).
Value generators are intended to be interpreted as an \emph{initial state}:
in the first cycle of execution they will emit their value, and produce the
disconnected \(\bullet\) value after that.
This is why there is no \(\bullet\) value generator; instead it is a
\emph{combinational} generator \(
\iltikzfig{strings/structure/monoid/init}[colour=comb]
\) intended to always produce the \(\bullet\) value.

\begin{nota}
    Although \(
    \iltikzfig{strings/structure/monoid/init}[colour=comb]
    \) is itself not a sequential value, when we refer to an arbitrary value
    \(
    \iltikzfig{circuits/components/values/vs}[val=v]
    \), \(v\) can be any value in \(\values\) including \(\bullet\).
    For a word of values \(\listvar{v} \in \valuetuple{n}\) (again possibly
    including \(\bullet\)), we may draw multiple value generators collapsed into
    one as \(
    \iltikzfig{circuits/components/values/vs}[val=\listvar{v},width=n]
    \), defined inductively over \(\listvar{v}\) as
    \begin{gather*}
        \iltikzfig{circuits/components/values/vs}[val=\varepsilon,width=0]
        \coloneqq
        \iltikzfig{strings/monoidal/empty}
        \qquad
        \iltikzfig{circuits/components/values/vs-even-longer}[val=v\listvar{w},width=n+1]
        \coloneqq
        \iltikzfig{circuits/components/values/vs-construction}
    \end{gather*}
\end{nota}

\begin{exa}
    The `values' of \(\scirc{\belnapsignature}\) are \(
    \iltikzfig{strings/structure/monoid/init}[colour=comb]
    \), \(
    \iltikzfig{circuits/components/values/vs}[val=\belnapfalse]
    \), \(
    \iltikzfig{circuits/components/values/vs}[val=\belnaptrue]
    \), \(
    \iltikzfig{circuits/components/values/vs}[val=\top]
    \); the first is a combinational generator and the latter three are
    sequential.
\end{exa}

The delay component is the opposite of a value; in the first cycle of execution
it is intended to produce the \(\bullet\) value, but in future cycles it outputs
the signal it received in the previous cycle.

\begin{rem}
    While the mathematical interpretation of a delay is straightforward, the
    physical aspect of a digital circuit it models is less clear.
    An obvious interpretation could be that the delay models a D flipflop in
    a clocked circuit, so the delay is one clock cycle.
    A more subtle interpretation is the `minimum obervable duration'; in this
    case the delay models inertial delay on wires up to some fixed precision.
\end{rem}

\begin{nota}
    Like values, we can derive delay components for arbitrary-bit wires, drawn
    like \(
    \iltikzfig{circuits/components/waveforms/delay}[width=n]
    \).
    \begin{gather*}
        \iltikzfig{circuits/components/waveforms/unit-delay}[width=0]
        \coloneqq
        \iltikzfig{strings/monoidal/empty}
        \qquad
        \iltikzfig{circuits/components/waveforms/delay}[width=n+1]
        \coloneqq
        \iltikzfig{circuits/components/waveforms/delay-construction}
    \end{gather*}
\end{nota}

Often one may also want to think about delays with some explicit `initial
value', like a sort of register.
This is so common that we introduce special notation for it.

\begin{nota}[Register]\label{not:register}
    For a word \(\listvar{v} \in \valuetuple{m}\), let \(
    \iltikzfig{circuits/components/waveforms/register}
    \coloneqq
    \iltikzfig{circuits/components/waveforms/register-shorthand}
    \).
\end{nota}

To distinguish them from combinational circuits, arbitrary sequential circuit
morphisms are drawn as green boxes \(
\iltikzfig{strings/category/f}[box=f,colour=seq,dom=m,cod=n]
\).

\begin{exa}[SR latch]\label{ex:sr-latch}
    A sequential circuit one might come across early on in an electronics
    textbook is the \emph{SR NOR latch}, one of the simplest registers.
    A possible design and interpretation in \(\scirc{\belnapsignature}\) are
    illustrated below.
    \begin{gather*}
        \iltikzfig{circuits/examples/sr-latch/real-circuit}
        \qquad
        \iltikzfig{circuits/examples/sr-latch/circuit}
    \end{gather*}

    SR NOR latches are used to hold state.
    They have two inputs: Reset (\(\mathsf{R}\)) and Set (\(\mathsf{S}\)), and
    two outputs \(\mathsf{Q}\) and \(\overline{\mathsf{Q}}\) which are always
    negations of each other.
    When Set receives a true signal, the \(\mathsf{Q}\) output is forced true,
    and will remain as such even if the Set input stops being pulsed true.
    It is only when the Reset input is pulsed true that the \(\mathsf{Q}\)
    output will return to false.
    (It is illegal for both Set and Reset to be pulsed high simultaneously; this
    issue is fixed in more complicated latches).

    SR latches work because of delays in how gates and wires transmit signals;
    one of the feedback loops between the two \(\norgate\) gates will `win'.
    We can model this in \(\scircsigma\) by using a different number of delay
    generators between the top and the bottom of the latch.
    We have opted for just the one because otherwise the upcoming examples
    become excessively complicated, but any number would do, so long as the top
    and bottom differ.
\end{exa}

\subsection{Generalised circuit signatures}

In a circuit signature, gates are assigned a number of input and output wires.
This serves us well when we want to model lower level circuits in which we
really are dealing with single-bit wires.
However, when designing circuits it is often advantageous to work at a higher
level of abstraction with `thicker' wires carrying words of information.
For example, the values in the circuits could be used to represent binary
numbers.

This can still be modelled in \(\scircsigma\) `as is' by using lots of parallel
wires to connect to the various primitives, but this can get messy with wires
all over the place.
Instead, we will introduce a generalisation of circuit signatures in which these
thicker buses of wires are treated as first-class objects.

\begin{defi}[Generalised circuit signature]
    A \emph{generalised circuit signature} \(\circuitsignature\) is a tuple \((
    \values,
    \disconnected,
    \circuitsignaturegates,
    \circuitsignaturearity,
    \circuitsignaturecoarity
    )\) where \(\values\) is a finite set of values, \(
    \disconnected \in \values
    \) is a \emph{disconnected} value, \(\circuitsignaturegates\) is a (usually
    finite) set of \emph{gate symbols}, \(
    \morph{\circuitsignaturearity}{\circuitsignaturegates}{\natplus^\star}
    \) is an \emph{arity} function and \(
    \morph{\circuitsignaturecoarity}{\circuitsignaturegates}{\natplus^\star}
    \) is a \emph{coarity} function.
\end{defi}

In a generalised circuit signature, primitives are typed with input and output
\emph{words} rather than just natural numbers.

\begin{exa}
    The generalised circuit signature for \emph{simple arithmetic circuits} is
    \(
    \belnapsignature^+ \coloneqq \left(
    \belnapvalues,
    \bot,
    \belnapgates^+,
    \belnaparity^+,
    \belnapcoarity^+
    \right)
    \), where \begin{gather*}
        \belnapgates
        \coloneqq \{
        \andgate_{k,n},
        \orgate_{k,n},
        \notgate_{n},
        \mathsf{add}_n
        \,|\,
        n \in \natplus
        \}
        \\[0.5em]
        \belnaparity^+(\andgate_{k,n}) \coloneqq [n \,|\, i < k]
        \quad
        \belnaparity^+(\orgate_{k,n}) \coloneqq [n \,|\, i < k]
        \\[0.5em]
        \belnaparity^+(\notgate_{n}) \coloneqq [n]
        \quad
        \belnaparity^+(\mathsf{add}_n) \coloneqq [n,n]
        \\[0.5em]
        \belnapcoarity^+
        \coloneqq
        \andgate_{k,n} \mapsto [n],
        \orgate_{k,n} \mapsto [n],
        \notgate_n \mapsto [n],
        \mathsf{add}_n \mapsto [n]
    \end{gather*}
    The gates \(\andgate_{k,n}\) and \(\orgate_{k,n}\) are gates that operate
    on \(k\) input wires of width \(n\); similarly the \(\notgate_n\) gate
    operates on input wires of width \(n\).
    The \(\mathsf{add}_n\) component represents an adder that takes as input
    two \(n\)-bit wires and outputs their \(n\)-bit sum.
\end{exa}

Just like a monochromatic circuit signature generates monochromatic PROPs, a
generalised circuit signature generates \(\natplus\)-coloured PROPs.

\begin{defi}
    For a generalised circuit signature \(\circuitsignature\), let the set
    \(\generators[\ccirc{}^+]\) of
    \emph{generalised combinational circuit generators} be defined as the set
    containing
    \begin{gather*}
        \iltikzfig{circuits/components/gates/gate}[gate=g,colour=comb,dom=\circuitsignaturearity(g),cod=\circuitsignaturecoarity(g)]
        \,
        \text{for each}\ g \in \circuitsignaturegates
        \\[1em]
        \iltikzfig{strings/structure/monoid/init}[colour=comb,obj=n]
        \quad
        \iltikzfig{strings/structure/comonoid/copy}[colour=comb,obj=n]
        \quad
        \iltikzfig{strings/structure/monoid/merge}[colour=comb, obj=n]
        \quad
        \iltikzfig{strings/structure/comonoid/discard}[colour=comb,obj=n]
        \quad
        \iltikzfig{strings/strictifiers/expand}[colour=comb,obj=n]
        \quad
        \iltikzfig{strings/strictifiers/collapse}[colour=comb,obj=n]
        \quad
        \text{for each}\ n \in \natplus
    \end{gather*}
    We write \(\ccircsigmag\) for the freely generated \(\natplus\)-coloured
    PROP \(\smc{\generators[\ccirc{}^+]}\).
\end{defi}

Most of the generators  in \(\ccircsigmag\) are fairly straightforward
generalisations of the primitives in \(\ccircsigma\) to act on each
colour (width) of wires.
The only new generators are the \emph{bundlers} at the end of the bottom row;
their intended meaning is that they can be used to \emph{split} and
\emph{combine} bundles of wire buses into bundles with different widths.
These constructs were first proposed by Wilson et al in~\cite{wilson2023string}
as a notation for \emph{non-strict categories}.
We take inspiration from their observation that a similar idea could also be
applied to strict categories.

\begin{exa}[ALU]
    The computation of a CPU is performed by an \emph{arithmetic logic unit},
    or ALU for short.
    An ALU takes some input wires of a fixed width and performs an operation
    on them given some control signal.
    While ALUs can often perform many different operations, we will look at an
    example operating on four-bit wires that performs a bitwise \(\andgate\)
    operation when the control is false, and an addition when the control is
    true.
    This ALU will also produce an output indicating if the sum is zero, and
    the sign of the sum; these auxiliary outputs only produce useful output when
    the addition operation is selected.

    \begin{center}
        \iltikzfig{circuits/examples/alu}
    \end{center}

    To apply the single-bit control to the four-bit \(\andgate\) gates, the
    top bundler and forks are used to create a wire containing only the
    original bit.

    The sum is zero if all of the bits are false.
    To determine this, the \(\orgate_{1,4}\) gate folds the four-bit sum into
    a one-bit value which is true if at least one of the bits is true.
    The \(\notgate_{1,1}\) inverts the output to produce true if there are no
    true bits.

    In two's complement representation, the most significant bit indicates if
    the sum is negative.
    To model this, the lower bundler splits the four-bit sum into its
    constituent bits, discarding the least significant three.
\end{exa}

Sequential circuits are generalised in the same way.

\begin{defi}
    For a generalised circuit signature \(\circuitsignature\), let the set
    \(\generators[\scirc{}^+]\) of
    \emph{generalised sequential circuit generators} be the set of
    generalised combinational circuit generators along with
    \(
    \iltikzfig{circuits/components/values/vs}[val=\listvar{v},width=n]
    \) for each \(n \in \natplus\) and \(\listvar{v} \in \valuetuple{n}\), and
    \(
    \iltikzfig{circuits/components/waveforms/delay}[width=n]
    \) for each \(n \in \natplus\).
    We write \(\scircsigmag\) for the freely generated PROP
    \(\stmc{\generators[\scirc{}^+]}\).
\end{defi}

Most of the upcoming results will be shown for the monochromatic case, as the
proofs are more elegant.
However, most of the results also generalise to the coloured case, and this will
be remarked upon throughout.

\section{Denotational semantics}

Circuits in \(\scircsigma\) are purely \emph{syntax}: they currently have no
behaviour associated with them.
In this section we will present a fully compositional denotational
semantics for sequential circuits based on \emph{causal}, \emph{monotone} and
\emph{finitely specified} stream functions.
This denotational semantics is constructed \emph{compositionally} using a
functor from \(\scircsigma\) to a PROP of stream functions with the desired
properties; a reverse functor is then defined which maps a stream function \(f\)
to a circuit in \(\scircsigma\) with \(f\) as its denotation, showing that this
denotational semantics is \emph{sound and complete}.

\begin{rem}
    In \cite{mendler2012constructive}, the semantics of digital circuits with
    delays cycles are presented using \emph{timed ternary simulation}, an
    algorithm to compute how a sequence of circuit outputs stabilises over time
    given the inputs and value of the current state.
    Essentially, one must solve a system of equations in terms of the nodes
    inside a circuit to determine its behaviour.
    Our approach is different as we assign each circuit a concrete stream
    function describing its behaviour, and show that every such stream function
    is the behaviour of at least one circuit in \(\scircsigma\).
\end{rem}

We will interpret digital circuits as a certain class of
\emph{stream functions}, functions that operate on infinite sequences of values.
This represents how the output of a circuit may not just operate on the current
input, but all of the previous ones as well.

\begin{rem}
    In \cite{mendler2012constructive}, the semantics of digital circuits with
    delays and cycles are presented using \emph{timed ternary simulation}, an
    algorithm to compute how a sequence of circuit outputs stabilises over time
    given the inputs and value of the current state.
    This differs from our approach as we assign each circuit a concrete stream
    function describing its behaviour, rather than having to solve a system of
    equations in terms of the gates inside a circuit in order to determine its
    behaviour.
\end{rem}

Recall that a function \(f\) between two posets is \emph{monotone}
if \(x \leq y \Rightarrow f(x) \leq f(y)\), and a \emph{lattice} is a poset
in which each pair of elements has a least upper bound \(\vee\) (a \emph{join})
and a greatest lower bound \(\wedge\) (a \emph{meet}); subsequently
every \emph{finite} lattice has an \emph{infimum} \(\bot\) and a \emph{supremum}
\(\top\).
We write \(v^n\) for the \(n\)-tuple containing only \(v\), and call a function
\(\morph{f}{\valuetuple{m}}{\valuetuple{n}}\) \emph{\(\bot\)-preserving} if
\(f(\bot^m) = \bot^n\).

\subsection{Interpreting circuit components}\label{sec:interpreting-components}

Before assigning stream functions to a given circuit in \(\scircsigma\), we will
first decide how to interpret the individual \emph{components} of a given
circuit signature.
First we consider the interpretation of the \emph{values} that flow through the
wires in our circuits.
In the denotational semantics the set of values will need to have a bit more
structure, as it must be ordered by how much \emph{information} each value
carries.

In our context of digital circuits the least and greatest elements respectively
represent signals with a complete \emph{lack} of information and \emph{every}
piece of information at once.
However, we need to add more structure than just this; given two signals we want
a way to be able to combine their information into one signal.

\begin{defi}
    Let \(A,B\) be finite lattices, where \(\bot_A\) is the least element of
    \(A\) and \(\bot_B\) is the least element of \(B\).
    A function \(\morph{f}{A}{B}\) is \emph{\(\bot\)-preserving} if
    \(f(\bot_A) = \bot_B\).
\end{defi}

Assigning interpretations to the combinational components of a circuit sets the
stage for the entire denotational semantics.

\begin{defi}[Interpretation]
    For a signature \(
    \signature = (
    \values, \bullet, \circuitsignaturegates, \circuitsignaturearity,
    \circuitsignaturecoarity
    )\), an \emph{interpretation} of
    \(\signature\) is a tuple \((\sqsubseteq, \gateinterpretation)\) where
    \((\values, \sqsubseteq)\) is a lattice with \(\bullet\) as the least
    element, and \(\gateinterpretation\) maps each
    \(p \in \circuitsignaturegates\) to a \(\bot\)-preserving monotone function
    \(
    \valuetuple{\circuitsignaturearity(p)}
    \to
    \valuetuple{\circuitsignaturecoarity(p)}
    \).
\end{defi}

\begin{exa}\label{ex:belnap-interpretation}
    \index{Belnap!interpretation}
    Recall the Belnap signature \(
    \belnapsignature = (
    \belnapvalues, \bot, \belnapgates, \belnaparity, \belnapcoarity
    )
    \) from \cref{ex:belnap-signature}.
    We assign a partial order \(\leq_\mathsf{B}\) to values in
    \(\belnapvalues\) as follows:

    \begin{center}
        \begin{tikzcd}
            & \top & \\
            \belnapfalse \arrow[dash]{ur} & & \belnaptrue \arrow[dash]{ul} \\
            & \bot \arrow[dash]{ul} \arrow[dash]{ur} &
        \end{tikzcd}
    \end{center}

    The gate interpretation \(\belnapgateinterpretation\) has action \(
    \andgate \mapsto \land, \orgate \mapsto \lor, \notgate \mapsto \neg
    \) where \(\land\), \(\lor\) and \(\neg\) are defined by the following
    truth tables~\cite{belnap1977useful}:

    \begin{center}
        \begin{tabular}{|c|cccc|}
            \hline
            \(\land\)        & \(\bot\)         & \(\belnapfalse\) & \(\belnaptrue\)  & \(\top\)         \\
            \hline
            \(\bot\)         & \(\bot\)         & \(\belnapfalse\) & \(\bot\)         & \(\belnapfalse\) \\
            \(\belnapfalse\) & \(\belnapfalse\) & \(\belnapfalse\) & \(\belnapfalse\) & \(\belnapfalse\) \\
            \(\belnaptrue\)  & \(\bot\)         & \(\belnapfalse\) & \(\belnaptrue\)  & \(\top\)         \\
            \(\top\)         & \(\belnapfalse\) & \(\belnapfalse\) & \(\top\)         & \(\top\)         \\
            \hline
        \end{tabular}
        \quad
        \begin{tabular}{|c|cccc|}
            \hline
            \(\lor\)         & \(\bot\)        & \(\belnapfalse\) & \(\belnaptrue\) & \(\top\)        \\
            \hline
            \(\bot\)         & \(\bot\)        & \(\bot\)         & \(\belnaptrue\) & \(\belnaptrue\) \\
            \(\belnapfalse\) & \(\bot\)        & \(\belnapfalse\) & \(\belnaptrue\) & \(\top\)        \\
            \(\belnaptrue\)  & \(\belnaptrue\) & \(\belnaptrue\)  & \(\belnaptrue\) & \(\belnaptrue\) \\
            \(\top\)         & \(\belnaptrue\) & \(\top\)         & \(\belnaptrue\) & \(\top\)        \\
            \hline
        \end{tabular}
        \quad
        \begin{tabular}{|c|c|}
            \hline
            \(\neg\)         &                  \\
            \hline
            \(\bot\)         & \(\bot\)         \\
            \(\belnaptrue\)  & \(\belnapfalse\) \\
            \(\belnapfalse\) & \(\belnaptrue\)  \\
            \(\top\)         & \(\top\)         \\
            \hline
        \end{tabular}
    \end{center}

    The Belnap interpretation is then \(
    (\leq_\mathsf{B}, \belnapgateinterpretation)
    \).
    An online tool for experimenting with the Belnap interpretation can be found
    at \url{https://belnap.georgejkaye.com}.
\end{exa}
\subsection{Denotational semantics of combinational circuits}

The semantic domain for \emph{combinational} circuits is straightforward: each
circuit maps to a monotone function.

\begin{defi}
    Let \(\funci\) be the PROP in which the morphisms
    \(m \to n\) are the \(\bot\)-preserving monotone
    functions \(\valuetuple{m} \to \valuetuple{n}\).
\end{defi}

To map between PROPs we must use a PROP morphism.

\begin{defi}
    Let \(\morph{\circuittofunci}{\ccircsigma}{\funci}\) be the PROP morphism
    with action defined as%
    \vspace{-\abovedisplayskip}
    \begin{center}
        \begin{minipage}{0.32\textwidth}
            \centering
            \begin{align*}
                \circuittofunci[
                    \iltikzfig{strings/structure/comonoid/copy}[colour=comb]
                ]
                 & \coloneqq
                (v) \mapsto (v, v)
                \\
                \circuittofunci[
                    \iltikzfig{strings/structure/monoid/merge}[colour=comb]
                ]
                 & \coloneqq
                (v, w) \mapsto v \sqcup w
            \end{align*}
        \end{minipage}
        \quad
        \begin{minipage}{0.25\textwidth}
            \centering
            \begin{align*}
                \circuittofunci[
                    \iltikzfig{strings/structure/comonoid/discard}[colour=comb]
                ]
                 & \coloneqq
                (v) \mapsto ()
                \\
                \circuittofunci[
                    \iltikzfig{strings/structure/monoid/init}[colour=comb]
                ]
                 & \coloneqq
                () \mapsto \bot
            \end{align*}
        \end{minipage}
        \quad
        \begin{minipage}{0.25\textwidth}
            \centering
            \vspace{1.5em}
            \(\circuittofunci[
                \iltikzfig{circuits/components/gates/gate}[gate=p,dom=m,cod=n]
            ]
            \coloneqq
            \gateinterpretation[p]
            \)
        \end{minipage}
    \end{center}
\end{defi}

\begin{rem}
    One might wonder why the fork and the join have different semantics, as they
    would be physically realised by the same wiring.
    This is because digital circuits have a notion of \emph{static causality}:
    outputs can only connect to inputs.
    This is why the semantics of combinational circuits is \emph{functions} and
    not \emph{relations}.

    In real life one could force together two digital devices, but this might
    lead to undefined behaviour in the digital realm.
    This is reflected in the semantics by the use of the join; for example, in
    the Belnap interpretation if one tries to join together \(\belnaptrue\) and
    \(\belnapfalse\) then the overspecified \(\top\) value is produced.
\end{rem}

\subsection{Denotational semantics of sequential circuits}

As one might expect, sequential circuits are slightly trickier to deal with.
In a combinational circuit, the output only depends on the inputs at the current
cycle, but for sequential circuits inputs can affect outputs many cycles after
they occur.

We therefore have to reason with \emph{infinite sequences} of inputs rather than
individual values; these are known as \emph{streams}.

\begin{nota}
    Given a set \(A\), we denote the set of streams (infinite sequences) of
    \(A\) by \(\stream{A}\).
    As a stream can equivalently be viewed as a function \(\nat \to A\), we
    write \(\sigma(i) \in A\) for the \(i\)-th element of stream
    \(\sigma \in \stream{A}\).
    Individual streams are written as \(
    \sigma \in \stream{A}
    \coloneqq
    \sigma(0) \streamcons \sigma(1) \streamcons
    \sigma(2) \streamcons \cdots
    \).
\end{nota}

Streams can be viewed a bit like lists, in that they have a head component and
an (infinite) tail component.

\begin{defi}[Operations on streams]\label{def:stream-operations}
    Given a stream \(\sigma \in \stream{A}\), its \emph{initial value}
    \(\streaminit(\sigma) \in A\) is defined as \(\sigma \mapsto \sigma(0)\)
    and its \emph{stream derivative} \(\streamderv(\sigma) \in \stream{A}\) is
    defined as \(\sigma \mapsto (i \mapsto \sigma(i+1))\).
\end{defi}

\begin{nota}
    For a stream \(\sigma\) with initial value \(a\) and stream derivative
    \(\tau\) we write it as \(\sigma \coloneqq a \streamcons \tau\).
\end{nota}

Streams will serve as the inputs and outputs to circuits, so the denotations of
sequential circuits will be \emph{stream functions}, which consume and produce
streams.
Just like with functions, we cannot claim that all streams are the
denotations of sequential circuits.

\begin{defi}[Causal stream function~\cite{rutten2006algebraic}]
    A stream function \(\morph{f}{\stream{A}}{\stream{B}}\) is \emph{causal} if,
    for all \(i \in \nat\) and \(\sigma,\tau \in \stream{A}\) we have that
    \(\sigma(j) = \tau(j)\) for all \(j \leq i\), then
    \(f(\sigma)(i) = f(\tau)(i)\).
\end{defi}

Causality is a form of continuity; a causal stream function is a stream function
in which the \(i\)-th element of its output stream only depends on elements
\(0\) through \(i\) inclusive of the input stream; it cannot look into the
future.
A neat consequence of causality is that it enables us to lift the stream
operations of initial value and stream derivative to stream \emph{functions}.

\begin{defi}[Initial output~\cite{rutten2006algebraic}]
    For a causal stream function \(\morph{f}{\stream{A}}{\stream{B}}\) and
    \(a \in A\), the \emph{initial output of \(f\) on input \(a\)} is an element
    \(\initialoutput{f}{a} \in A\) defined as
    \(\initialoutput{f}{a} \coloneqq \streaminit(f(a \streamcons \sigma))\) for
    an arbitrary \(\sigma \in \stream{A}\).
\end{defi}

Since \(f\) is causal, the stream \(\sigma\) in the definition of initial
output truly is arbitrary; the \(\streaminit\) function only depends on the
first element of the stream.

\begin{defi}[Functional stream derivative~\cite{rutten2006algebraic}]
    For a stream function \(\morph{f}{\stream{A}}{\stream{B}}\) and
    \(a \in A\), the
    \emph{functional stream derivative of \(f\) on input \(a\)} is a stream
    function \(\streamderivative{f}{a}\) defined as \(
    \streamderivative{f}{a}
    \coloneqq
    \sigma \mapsto \streamderv(f(a \streamcons \sigma))
    \).
\end{defi}

The functional stream derivative \(\streamderivative{f}{a}\) is a new stream
function which acts as \(f\) would `had it seen \(a\) first'.

\begin{rem}
    One intuitive way to view stream functions is to think of them as the states
    of some Mealy machine; the initial output is the output given some input,
    and the functional stream derivative is the transition to a new state.
    As with most things in mathematics, this is no coincidence; there is a
    homomorphism from any Mealy machine to a stream function.
    We will exploit this fact in the next section.
\end{rem}

This leads us to the next property of denotations of sequential circuits.
Although they may have infinitely many inputs and outputs, circuits themselves
are built from a finite number of components.
This means they cannot specify infinite \emph{behaviour}.

\begin{nota}
    Given a finite word \(\listvar{a} \in \freemon{A}\), we abuse notation
    and write \(\streamderivative{f}{\listvar{a}}\) for the repeated
    application of the functional stream derivative for the elements of
    \(\listvar{a}\), i.e.\ \(
    \streamderivative{f}{\varepsilon} \coloneqq f
    \) and \(
    \streamderivative{f}{a \streamcons \listvar{b}} \coloneqq
    \streamderivative{(\streamderivative{f}{a})}{\listvar{b}}
    \).
\end{nota}

\begin{defi}
    Given a stream function \(\morph{f}{\stream{A}}{\stream{B}}\), we say it is
    \emph{finitely specified} if the set \(\{
    \streamderivative{f}{\listvar{a}} \,|\, \listvar{a} \in \freemon{A}
    \}\) is finite.
\end{defi}

As the components of our circuits are monotone and \(\bot\)-preserving, a
denotational semantics for circuits must also be monotone and
\(\bot\)-preserving.
This means we need to lift the order on values to an order on streams.

\begin{nota}
    For a poset \((A, \leq_A)\) and streams \(\sigma,\tau \in \stream{A}\), we
    say that \(\sigma \leq_{\stream{A}} \tau\) if \(\sigma(i) \leq_A \tau(i)\)
    for all \(i \in \nat\).
\end{nota}

For these properties to be suitable as a denotational semantics for
sequential circuits, we must show that stream functions with these
properties form a category we can map into from \(\scircsigma\).
We will first show that these categories form a symmetric monoidal category, so
we need a suitable candidate for composition and tensor.
There are fairly obvious choices here: for the former we use regular function
composition and for the latter we use the Cartesian product.

\begin{lem}\label{lem:causality-preserved}
    Causality is preserved by composition and Cartesian product.
\end{lem}
\begin{proof}
    If the \(i\)-th element of two stream functions \(f\) and \(g\) only depends
    on the first \(i+1\) elements of the input, then so will their composition
    and product.
\end{proof}

\begin{lem}\label{lem:finitely-specified-preserved}
    Finite specification is preserved by composition and Cartesian
    product.
\end{lem}
\begin{proof}
    For both the composition and product of two stream functions \(f\) and
    \(g\), the largest the set of stream derivatives could be is the product of
    stream derivatives of \(f\) and \(g\), so this will also be finite.
\end{proof}

\begin{lem}\label{lem:monotonicity-preserved}
    \(\bot\)-preserving monotonicity is preserved by composition and Cartesian
    product.
\end{lem}
\begin{proof}
    The composition and product of any monotone function is monotone, and must
    preserve the \(\bot\) element.
\end{proof}

As the categorical operations preserve the desired properties, these stream
functions form a PROP.

\begin{prop}
    There is a PROP \(\streami\) in which the morphisms \(m \to n\) are the
    causal, finitely specified and \(\bot\)-preserving monotone stream
    functions \(\valuetuplestream{m} \to \valuetuplestream{n}\).
\end{prop}
\begin{proof}
    Identity is the identity function, the symmetry swaps streams, composition
    is composition of functions, and tensor product on morphisms
    \(\morph{f}{\valuetuplestream{m}}{\valuetuplestream{n}}\) and
    \(\morph{g}{\valuetuplestream{p}}{\valuetuplestream{q}}\) is the Cartesian
    product of functions composed with the components of the isomorphism
    \(\valuetuplestream{m} \times \valuetuplestream{n}
    \cong \valuetuplestream{m+n}\).

    As these constructs satisfy the categorical axioms, and as function
    composition and Cartesian product preserve causality
    (\cref{lem:causality-preserved}),
    finite specification (\cref{lem:finitely-specified-preserved}),
    and monotonicity (\cref{lem:monotonicity-preserved}), this data defines a
    symmetric monoidal category.
\end{proof}

Modelling sequential circuits as stream functions deals with temporal
aspects, but what about feedback?
As the assignment of denotations needs to be compositional, we need
to map the trace on \(\scircsigma\) to a trace on \(\streami\).
A usual candidate for the trace when considering partially ordered settings is
the \emph{least fixed point}.

\begin{defi}[Least fixed point]
    For a poset \((A, \leq)\) and function \(\morph{f}{A}{A}\), the least
    fixed point of \(f\) is a value \(\mu_f\) such that \(f(\mu_f) = f\) and,
    for all values \(v\) such that \(f(v) = v\), \(\mu_f \leq v\).
\end{defi}

Least fixed points are ubiquitous in program semantics, where they are often
used to model \emph{recursion}; since feedback is a form of recursion it seems
apt that we should also follow this route.
As fixed points are so important, they are the subject of many theorems; one
that will come in very useful for us is the \emph{Kleene fixed-point thm},
which is concerned with a special class of monotone functions.

\begin{nota}[Image]
    For a function \(\morph{f}{A}{B}\) and subset \(C \subseteq A\), we write
    \(f[C] \subseteq B\) for the \emph{image} of \(C\) under \(f\).
\end{nota}

\begin{defi}[Scott continuity]
    Given two posets \((A, \leq_A)\) and \((B, \leq_B)\), a function
    \(\morph{f}{A}{B}\) is \emph{Scott-continuous} if for every directed
    subset \(C \subseteq A\) it holds that \(f(\bigvee_A C) = \bigvee_B(f[C])\)
    i.e.\ \(f\) preserves directed joins.
\end{defi}

\begin{thm}[Kleene fixed-point theorem~\cite{tarski1955latticetheoretical}]
    \index{Kleene fixed-point thm}
    Let \((A, \leq)\) be a poset in which each of its directed subsets has a
    join, and let \(\morph{f}{L}{L}\) be a Scott-continuous function.
    Then \(f\) has a least fixed point, defined as \(
    \bot \vee f(\bot) \vee f(f(\bot)) \vee \cdots
    \).
\end{thm}

So far we have not explicitly enforced Scott-continuity on stream functions; it
turns out that it is implied by causality and monotonicity.

\begin{prop}\label{prop:monotone-causal-scott}
    Let \((A, \leq_A)\) and \((B, \leq_B)\) be finite lattices, and let
    \((\stream{A}, \leq_{\stream{A}})\) and \((\stream{B}, \leq_{\stream{B}})\)
    be the induced lattices on streams.
    Then a causal and monotone function \(\stream{A} \to \stream{B}\) must also
    be Scott-continuous.
\end{prop}
\begin{proof}
    Consider a directed subset \(D \subseteq \stream{A}\); we need to show that
    for an arbitrary causal, monotone and finitely specified function \(f\) we
    have that \(f\left(\bigvee D\right) = \bigvee f[D]\).

    First consider the case when there is a greatest element in \(D\).
    In this case, \(\bigvee D\) must be the greatest element, and as such
    \(\bigvee D \in D\).
    As \(f\) is monotone then \(f(\bigvee D)\) must be the greatest element in
    \(f[D]\); subsequently, \(f\left(\bigvee D\right) = \bigvee f[D]\).

    Now consider the case where there is no greatest element in \(D\) and
    subsequently \(\bigvee D \not\in D\); if there is no greatest element,
    \(D\) must be infinite.
    Even though it is infinite, \(D\) is a directed subset so each pair of
    elements must have an upper bound, and as \(\leq_{\stream{A}}\) is computed
    pointwise by using \(\leq_A\) we can consider what the upper bounds are
    pointwise too.
    Because \(A\) is finite, there cannot be an infinite chain of upper bounds
    for each element \(i\); there must exist an element \(a_i \in A\) such that
    \(D\) contains infinitely many streams \(\sigma\) such that
    \(\sigma(i) = a_i\).
    This means that \(\left(\bigvee D\right)(i) = a\), so every prefix of
    \(\bigvee D\) must exist as a prefix of a stream in \(D\).
    As \(f\) is causal, for each prefix
    \(f\left(\bigvee D\right)\) there must also exist a \(d \in D\) such that
    \(f(d)\) has that prefix, and as such
    \(\bigvee f[D] = f\left(\bigvee D\right)\).
\end{proof}

This means we can use the Kleene fixed point theorem as a tool to show that the
least fixed point is a trace on \(\streami\).

\begin{nota}[Concatenation]
    For a set \(A\), \(\listvar{a} \in A^m\), and \(\listvar{b} \in A^n\), we
    write \(\listvar{a} \concat \listvar{b}\) for the \emph{concatenation}
    of \(\listvar{a}\) and \(\listvar{b}\): the tuple of length \(m+n\)
    containing the elements of \(\listvar{a}\) followed by the elements of
    \(\listvar{b}\).
\end{nota}

\begin{nota}[Projection]
    For a set \(A\), let \(\listvar{a} \in A^m\) and \(\listvar{b} \in A^n\).
    Then for their concatenation \(
    \listvar{c} \coloneqq \listvar{a} \concat \listvar{b} \in A^{m+n}\),
    we write \(\proj{0}(\listvar{c}) = \listvar{a}\) and
    \(\proj{1}(\listvar{c}) = \listvar{b}\) for the \emph{projections}.
\end{nota}

\begin{lem}\label{lem:lfp-stream-function}
    Given a morphism \(
    \morph{f}{\valuetuplestream{x+m}}{\valuetuplestream{x+n}}
    \in \streami
    \), and stream \(\sigma \in \valuetuplestream{m}\), the function \(
    \tau \mapsto \proj{0}\mleft(f(\tau,\sigma)\mright)
    \) has a least fixed point.
\end{lem}
\begin{proof}
    The function \(\tau \mapsto \proj{0}\mleft(f(\tau,\sigma)\mright)\) is
    causal and monotone because \(f\) and the projection function are
    causal and monotone, so it is Scott-continuous by
    \cref{prop:monotone-causal-scott}.
    By the Kleene fixed point theorem, this function has a least fixed point,
    defined as \(
    \proj{0}\mleft(f(\bot^\omega, \sigma)\mright) \ljoin
    \proj{0}\mleft(f(\proj{0}\mleft(f(\bot^\omega, \sigma)\mright), \sigma)\mright) \ljoin
    \cdots
    \).
\end{proof}

We must show that this notion of least fixed point is a trace on \(\streami\).
The first step is to show that taking the least fixed point of a stream function
in \(\streami\) produces another causal, finitely specified,
\(\bot\)-preserving, and monotone stream function.

\begin{defi}\label{def:state-order}
    Let \(A\) and \(B\) be posets and let
    \(\morph{f, g}{\stream{A}}{\stream{B}}\) be stream functions.
    We say \(f \stateorder g\) if \(f(\sigma) \leq_{\stream{B}} g(\sigma)\)
    for all \(\sigma \in \stream{A}\).
\end{defi}

\begin{thm}\label{thm:trace-well-defined}
    For a function \(\morph{f}{\valuetuplestream{x+m}}{\valuetuplestream{x+n}}\),
    let \(\mu_f(\sigma)\) be the least fixed point of the function \(
    \tau \mapsto \proj{0}\mleft(f(\tau,\sigma)\mright)
    \).
    Then, the stream function \(
    \sigma \mapsto \proj{1}\mleft(f(\mu_f(\sigma), \sigma)\mright)
    \) is in \(\streami\).
\end{thm}
\begin{proof}
    To show this, we need to prove that
    \(\sigma \mapsto \proj{1}\mleft(f(\mu_f(\sigma)\sigma)\mright)\) is in
    \(\streami\): it is causal, finitely specified, and \(\bot\)-preserving
    monotone.

    Since \(
    \morph{f}{\valuetuplestream{x+m}}{\valuetuplestream{x+n}}
    \) is a morphism of \(\streami\), it has finitely many stream derivatives.
    For each stream derivative \(\streamderivative{f}{\,\listvar{w}}\), let the
    function \(
    \morph{
        \widehat{\streamderivative{f}{\listvar{w}}}
    }{
        \valuetuplestream{x+m}
    }{
        \valuetuplestream{x}
    }
    \) be defined as \(
    \tau\sigma
    \mapsto
    \proj{0}(\streamderivative{f}{\listvar{w}}(\tau\sigma))
    \).
    Note that each of these functions are causal, \(\bot\)-preserving, and
    monotone, because they are constructed from pieces that are causal
    \(\bot\)-preserving and monotone.

    In particular, \(\mu_f(\sigma)\) is the least fixed point of
    \(\widehat{f_\varepsilon}\left((-)\sigma\right)\).
    Using the Kleene fixed point theorem, the least fixed point of
    \(\widehat{f}((-)\sigma)\) can be obtained by composing
    \(\widehat{f}((-)\sigma)\) repeatedly with itself.
    This means that \(
    \mu_f(\sigma)
    =
    \bigsqcup_{k \in \nat} \widehat{f^k}(\bot^\omega,\sigma)
    \) where \(\widehat{f^k}\) is the \(k\)-fold composition of \(f(-,\sigma)\)
    with itself, i.e.\ \(\widehat{f^0}(\tau\sigma) = \tau\) and \(
    \widehat{f^{k+1}}(\tau\sigma)
    =
    \widehat{f}\left(\left(\widehat{f^{k}}(\sigma, \tau)\right)\sigma\right)
    \).
    That the mapping \(\mu_f\) is causal and monotone is
    straightforward: each of the functions in the join is causal and monotone,
    and join preserves these properties.
    It remains to show that this mapping has finitely many stream derivatives.

    When equipped with \(\stateorder\), the set of functions
    \(\valuetuplestream{x+m} \to \valuetuplestream{x}\)
    is a poset, of which
    \(\{\widehat{f_w} \,|\, w \in (\valuetuple{x+m})^\star\}\)
    is a finite subset.
    Restricting the ordering \(\stateorder\) to this set yields a finite poset.
    Since this poset is finite, the set of strictly increasing sequences in this
    poset is also finite.
    We will now demonstrate a relationship between these sequences and stream
    derivatives of \(\mu_f\).

    Suppose \(
    S = \widehat{f_{\,\listvar{w_0}}} \prec \widehat{f_{\,\listvar{w_1}}} \prec
    \cdots \prec \widehat{f_{\,\listvar{w_{\ell-1}}}}
    \) is a strictly increasing sequence of length \(\ell\) in the set of stream
    functions \(
    \{\widehat{f_w} \,|\, w \in (\valuetuple{x+m})^\star\}
    \).
    We define a function \(
    \morph{g_S}{
        \valuetuplestream{m}
    }{
        \valuetuplestream{x}
    }
    \) as \(
    (\sigma) \mapsto \bigsqcup_{k \in \nat} g_k(\sigma)
    \) where \[
        g_k(\sigma) =
        \begin{cases}
            \bot^\omega                                                                & \text{ if } k = 0              \\
            \widehat{f_{\,\listvar{w_k}}}(\left(g_{k-1}(\sigma), \sigma\right))        & \text{ if } 1 \leq k \leq \ell \\
            \widehat{f_{\,\listvar{w_{\ell-1}}}}(\left(g_{k-1}(\sigma), \sigma\right)) & \text{ if } \ell < k
        \end{cases}.
    \]
    Let the set \(G \coloneqq \{
    g_S \,|\, S \text{ is a strictly increasing sequence}
    \}\).
    When \(S\) is set to the one-item sequence \(\widehat{f}\), \(g_S\) is
    \(\mu_f\), so \(\mu_f \in G\).
    As \(G\) is finite, this means that if \(G\) is closed under stream
    derivative, \(\mu_f\) has finitely many stream derivatives.
    Any element of \(G\) is either \(\bot^\omega\) or has the form \(
    \sigma
    \mapsto
    \widehat{\streamderivative{f}{\,\listvar{w}}}(g_k(\sigma), \sigma)
    \) for some \(\sigma \in \valuetuplestream{m}\) and
    \(k > 0\).
    As \(\bot^\omega\) is its own stream derivative, we need to show that
    applying stream derivative to the latter produces another element of \(G\).
    \begin{align*}
        \streamderivative{\sigma \mapsto \left(\widehat{\streamderivative{f}{\, \listvar{w}}}\left(g_{k-1}(\sigma), \sigma\right)\right)}{(a,b)}
         & = \sigma \mapsto \streamderv\left(\widehat{\streamderivative{f}{\, \listvar{w}}}((a,b) \streamcons \left(g_{k-1}(\sigma), \sigma\right))\right)              \\
         & = \sigma \mapsto \streamderv\left(\proj{0}\mleft(\streamderivative{f}{\, \listvar{w}}((a,b) \streamcons \left(g_{k-1}(\sigma), \sigma\right))\mright)\right) \\
         & = \sigma \mapsto \proj{0}\mleft(\streamderv\left(\streamderivative{f}{\, \listvar{w}}((a,b) \streamcons \left(g_{k-1}(\sigma), \sigma\right))\right)\mright) \\
         & = \sigma \mapsto \proj{0}\mleft(\streamderivative{\left(\streamderivative{f}{\, \listvar{w}}(g_{k-1}(\sigma), \sigma)\right)}{(a,b)}\mright)                 \\
         & = \sigma \mapsto \proj{0}\mleft(\streamderivative{f}{\, (a,b) \streamcons \listvar{w}}(g_{k-1}(\sigma), \sigma)\mright)                                      \\
         & = \sigma \mapsto \widehat{\streamderivative{f}{\, (a,b) \streamcons \listvar{w}}}(g_{k-1}(\sigma), \sigma)
    \end{align*}
    As \(\proj{0}\mleft(\streamderivative{f}{(a,b) \streamcons \listvar{w}}\mright)\)
    is in \(G\), the latter  is closed under stream derivative.
    Subsequently, \(\mu_f\) has finitely many stream derivatives.

    This means that all the components of
    \(\sigma \mapsto \proj{1}(f(\mu_f(\sigma), \sigma))\) are causal, monotone and
    finitely specified, and as these properties are preserved by composition,
    the composite must also have them, so
    \(\sigma \mapsto \proj{1}(f(\mu_f(\sigma), \sigma))\) is in \(\streami\).
\end{proof}

Even if \(\streami\) is closed under least fixed point, this does not mean that
it is a valid trace.
To verify this we must establish that the categorical axioms of the trace hold.

\begin{thm}
    A trace on \(\streami\) is given for a function \(
    \morph{f}{\valuetuplestream{x+m}}{\valuetuplestream{x+n}}
    \) by the stream function \(
    \sigma \mapsto \proj{1}(f(\mu_f(\sigma), \sigma))
    \), where \(\mu_f(\sigma)\) is the least fixed point of the function \(
    \tau \mapsto \proj{0}\mleft(f(\tau,\sigma)\mright)
    \) for fixed \(\sigma\).
\end{thm}
\begin{proof}
    By \cref{thm:trace-well-defined}, \(\streami\) is closed under taking the
    least fixed point, so we just need to show that the axioms of STMCs hold.
    Most of these follow in a straightforward way; the only interesting one is
    the sliding axiom.
    We need to show that for stream functions \(
    \morph{f}{\valuetuplestream{x+m}}{\valuetuplestream{y+n}}
    \) and \(
    \morph{g}{\valuetuplestream{y}}{\valuetuplestream{x}}
    \), we have that \(
    \trace{y}{(\tau, \sigma) \mapsto f(g(\tau), \sigma)}
    =
    \trace{x}{
        (\tau, \sigma)
        \mapsto
        g(\proj{0}\mleft(f(\tau, \sigma)\mright),
        \proj{1}\mleft(f(\tau, \sigma)\mright))
    }
    \).

    Let \(l \coloneqq (\tau, \sigma) \mapsto f(g(\tau), \sigma)\) and
    \(r \coloneqq (\tau, \sigma)
    \mapsto
    g(\proj{0}\mleft(f(\tau, \sigma)\mright),
    \proj{1}\mleft(f(\tau, \sigma)\mright))\); we must apply the candidate
    trace construction to both of these and check they are equal.
    For \(l\), the least fixed point of \(
    \tau \mapsto \proj{0}\mleft(f(g(\tau), \sigma)\mright)
    \) is \[
        \mu_l(\sigma) =
        \proj{0}\mleft(f(g(\bot^\omega), \sigma)\mright) \ljoin
        \proj{0}\mleft(f(g(\proj{0}\mleft(f(g(\bot^\omega), \sigma))\mright), \sigma)\mright) \ljoin
        \cdots.\]
    Plugging this into the candidate trace construction we have that \begin{align*}
         &
        \sigma \mapsto \proj{1}\mleft(f(g(\mu_l^l(\sigma)), \sigma)\mright)
        \\
         & \qquad=
        \sigma \mapsto \proj{1}\mleft(f(g(\proj{0}\mleft(f(g(\dots f(g(\proj{0}\mleft(f(g(\bot^\omega), \sigma)\mright)), \sigma)))\mright)), \sigma)\mright)
    \end{align*}
    For the right-hand side, the least fixed point of \(
    \tau \mapsto g(\proj{0}\mleft(f(\tau, \sigma)\mright))
    \) is \[
        \mu_r(\sigma) =
        g(\proj{0}\mleft(f(\bot^\omega, \sigma)\mright)) \ljoin
        g(\proj{0}\mleft(f(g(\proj{0}\mleft(f(g(\bot^\omega), \sigma)\mright)), \sigma)\mright)) \ljoin
        \cdots
    \]
    When plugged into the candidate trace construction this produces \begin{align*}
         & \sigma \mapsto \proj{1}\mleft(g(\proj{0}(f(\mu_r(\sigma), \sigma))), \proj{1}(f(\mu_r(\sigma), \sigma))\mright)
        \\
         & \qquad =
        \sigma \mapsto \proj{1}(f(\mu_\sigma^r, \sigma))
        \\
         & \qquad=
        \sigma \mapsto \proj{1}(f(
        g(\proj{0}\mleft(f(g(\dots f(g(\proj{0}\mleft(f(g(\bot^\omega), \sigma)\mright)), \sigma)\mright)), \sigma))))
        \\
         & \qquad=
        \sigma \mapsto \proj{1}(f(
        g(\proj{0}\mleft(f(g(\dots f(g(\proj{0}\mleft(f(\bot^\omega, \sigma)\mright)), \sigma)\mright)), \sigma))))
    \end{align*}
    Both the left-hand and the right-hand sides of the sliding equation are
    equal, so the construction is indeed a trace.
\end{proof}

We now have two traced PROPs: a \emph{syntactic} PROP of sequential circuit
terms \(\scircsigma\) and a \emph{semantic} PROP of causal, finitely
specified, monotone stream functions \(\streami\).
It would be straightforward to now define a map from circuits to these stream
functions; indeed, this is the strategy used in~\cite{ghica2024fully}.
Instead, we will first examine another structure with close links to both
circuits and stream functions; that of \emph{Mealy machines}.
The structure of Mealy machines will come in useful when considering the
\emph{completeness} of the denotational semantics.

\bibliographystyle{alphaurl}
\bibliography{refs/refs}

\end{document}