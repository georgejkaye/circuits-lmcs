\documentclass{article}

\usepackage[margin=2cm]{geometry}

\usepackage{todonotes}

\title{Response to reviewers' comments}
\author{George Kaye}

\begin{document}

\maketitle

We thank the reviewers for their comments, and have implemented the following
improvements to the paper.

\section{Improvements to proofs}

\begin{itemize}
      \item \textbf{Proposition 3.23}: Used proof strategy suggested by reviewer
      \item \textbf{Theorem 3.28}: Reworked presentation of proof significantly for clarity.
            Some discussion on the proof:
            \begin{itemize}
                  \item This was certainly one of the most surprising results to us,
                        and we don't expect an analogous result to hold in many other contexts.
                  \item In summary, the stream derivatives of the stages of the Kleene
                        construction for $f$ (i.e. derivatives of $\mu_f^i$) form an ordered chain in the function space
                        and these can be put in correspondence with chains of derivatives
                        of $f$. That $f$ has finitely many derivatives ensures there
                        are finitely many such chains of its derivatives, which in turn
                        ensures there are finitely many derivatives of $\mu_f$.
            \end{itemize}
      \item \textbf{Theorem 3.29}: Used proof strategy suggested by reviewer
      \item \textbf{Definition 3.38}: Adjusted definition to state that the state
            set \(S\) is a lattice (this also fixes issue in \textbf{Proposition 3.48})
      \item \textbf{Definition 3.46}: Fixed typo in definition of least fixed point
            of Mealy machine; add in necessary projections to make expressions
            typecheck
      \item \textbf{Proposition 3.48}: Fixed typo in definition
            of \(f^\prime\)
      \item \textbf{Lemma 3.68} (originally Lemma 3.66): Used \(\gamma_\leq\)
            function where appropriate
      \item \textbf{Proposition 3.79} (originally Proposition 3.76): Updated
            Mealy machines to use correct variables
      \item \textbf{Theorem 3.80} (originally Theorem 3.79): Fixed typos in
            expressions
      \item \textbf{Lemma 4.10}: Referenced correct fixed point
      \item \textbf{Lemma 5.13}: Reworked proof strategy in order to avoid false
            statement
            \begin{itemize}
                  \item No longer claim that any monotone Mealy homomorphism is
                        monotone
                  \item \todo[inline]{Sum up reworked proof}
            \end{itemize}
\end{itemize}

\section{Improvements to narrative}

\begin{itemize}
      \item Moved diagram showing interactions of PROPs to end of paper
      \item References to `Belnap signature' replaced with `gate-level' signature,
            and \textbf{Remark 2.3} added about historical relevance of Belnap
            added
      \item Term `monochromatic' dropped
      \item \textbf{Definition 3.5-3.6}: Added explicit mention of circuit
            signature \(\Sigma\) and interpretation
            \(\mathcal{I}\)
      \item \textbf{Remark 3.4}: discuss the choice between \(\leq\) and
            \(\sqsubseteq\)
      \item For discussing completeness of combinational circuit semantics,
            explicitly make reference to how we used \(\bot\)-preserving
            functions rather than all functions
      \item Extract \textbf{Definition 3.51} out of \textbf{Corollary 3.53}
            (originally Corollary 3.51)
      \item Explicitly mention what we mean by soundness and completeness
      \item Added \textbf{Remark 3.72} about how a monotone completion can be
            viewed as a Kan extension
      \item Added \textbf{Section 4.5} elaborating on how the
            operational semantics can be mechanised by interpreting circuits as
            hypergraphs, and clarifying that this is only a useful tool for
            implementation purposes rather than a core part of the framework
\end{itemize}

\section{Improvements to presentation}

\begin{itemize}
      \item Changed string diagram boxes representing composite morphisms to have
            rounded corners
      \item \textbf{Definition 3.51} (previously part of Corollary 3.51):
            Changed to using uncurried Mealy function where possible
            and added \textbf{Remark 3.35} to discuss the two forms
      \item Add trace equation names to Figure 2
      \item \textbf{Definition 4.11}: use \(i\) for the index of iteration
      \item \textbf{Definition 6.14}: use `more efficient' rather than 'logically
            equivalent'
\end{itemize}


\section{Typos}

In addition to the points raised above, the minor typos and phrasing issues
raised by the reviewers have also been remedied.

\end{document}