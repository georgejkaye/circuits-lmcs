\documentclass{article}

\usepackage[margin=2cm]{geometry}

\usepackage{todonotes}

\title{Response to reviewers' comments}
\author{George Kaye}

\begin{document}

\maketitle

We thank the reviewers for their comments, and have implemented the following
improvements to the paper.

\section{Improvements to proofs}

\begin{itemize}
      \item \textbf{Proposition 3.23}: used proof strategy suggested by reviewer
      \item \textbf{Theorem 3.28}: reworked proof
            \begin{itemize}
                  \item \todo[inline]{Sum up reworked proof}
            \end{itemize}
      \item \textbf{Theorem 3.29}: used proof strategy suggested by reviewer
      \item \textbf{Definition 3.38}: adjusted definition to state that the state
            set \(S\) is a poset with a minimal element
      \item \textbf{Definition 3.46}: fix typo in definition of least fixed point
            of Mealy machine; add in necessary projections to make expressions
            typecheck
      \item \textbf{Proposition 3.48}: reword proof to state that the Mealy
            machine \((S, f)\) has a least fixed point, and fix typo in definition
            of \(f^\prime\)
      \item \textbf{Lemma 3.67} (originally Lemma 3.66): use \(\gamma_\leq\)
            function where appropriate
      \item \textbf{Proposition 3.78} (originally Proposition 3.76): updated
            Mealy machines to use correct variables
      \item \textbf{Theorem 3.79} (originally Theorem 3.79): fixed typos in
            expressions
      \item \textbf{Lemma 4.10}: referenced correct fixed point
      \item \textbf{Lemma 5.13}: reworked proof strategy in order to avoid false
            statement
            \begin{itemize}
                  \item No longer show that any monotone Mealy homomorphism is monotone
                        (which is indeed false)
                  \item \todo[inline]{Sum up reworked proof}
            \end{itemize}
\end{itemize}

\section{Improvements to narrative}

\begin{itemize}
      \item Moved diagram showing interactions of PROPs to end of paper
      \item References to `Belnap signature' replaced with `gate-level' signature,
            and remark about historical relevance of Belnap added
      \item Term `monochromatic' dropped
      \item Add Remark 3.4 discussing the choice between \(\leq\) and \(\sqsubseteq\)
      \item For discussing completeness of combinational circuit semantics,
            explicitly make reference to how we used \(\bot\)-preserving
      \item Extract Definition 3.50 out of Corollary 3.52
            (originally Corollary 3.51)
      \item Explicitly mention what we mean by soundness and completeness
      \item Add Remark 3.72 about how a monotone completion can be viewed as a Kan
            extension
      \item \textbf{Section 4}: added section 4.5 elaborating on how the
            operational semantics can be mechanised by interpreting circuits as
            hypergraphs, and clarifying that this is only a useful tool for
            implementation purposes rather than a core part of the framework
\end{itemize}

\section{Improvements to presentation}

\begin{itemize}
      \item Changed string diagram boxes representing composite morphisms to have
            rounded corners
      \item Become consistent where possible about using uncurried Mealy function
            \begin{itemize}
                  \item \todo[inline]{Check we actually do this}
                  \item When discussing the Mealy coalgebra the curried form is
                        preferred as the functor concerned really is
                        \(S \to (S \times B)^A\)
            \end{itemize}
      \item Add trace equation names to Figure 2
      \item \textbf{Definition 4.11}: use \(i\) for the index of iteration
      \item \textbf{Definition 6.14}: use `more efficient' rather than 'logically
            equivalent'
\end{itemize}


\section{Typos}

In addition to the points raised above, the minor typos and phrasing issues
raised by the reviewers have also been remedied.

\end{document}